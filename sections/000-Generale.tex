\documentclass[../main.tex]{subfiles}

\graphicspath{{\subfix{../images/}}}

\begin{document}

\newpage


\section{Introduzione Generale}

Introduciamo ora in maniera del tutto generale la struttura e suddivisione del corso nelle sue varie parti andando ad evidenziare tutte le sue parti. In particolare seguono maggiori informazioni sulle varie categorie.

\subsection{Suddivisione del corso}

Il corso é strutturato come molti altri evidenziando tre diversi momenti lezioni, esercitazioni e laboratori. In particolare non esiste una suddivisione netta tra lezioni ed esercitazioni motivo per cui potrebbero essere presenti degli esercizi di esempio all'interno delle lezioni e non solo all'interno delle sezioni di esercitazione. Per quando riguarda invece i laboratori saranno presenti risolti nella loro interezza con annesso l'eventuale codice richiesto.

Infine facciamo notare come il materiale proposto all'interno di questo elaborato segua in maniera più o meno precisa la suddivisione del corso proposto dai docenti in aula.

\subsection{Materiale}

Questo elaborato si basa principalmente sul materiale fornito durante il corso, tra cui slide, guide di laboratorio, esercizi e temi d’esame. In particolare per alcuni approfondimenti e chiarimenti si sono utilizzate alcune fonti tra cui i seguenti libri di testo consigliati per il corso:

\begin{enumerate}
	\item[\textbf{-}] G. Calafiore, Elementi di Automatica, CLUT, Torino, 2004.
	\item[\textbf{-}] S. Chiaverini, F. Caccavale, L. Villani, L. Sciavicco, Fondamenti di sistemi dinamici, McGraw-Hill, Milano, 2003.
	\item[\textbf{-}] P. Bolzern, R. Scattolini, N. Schiavoni, Fondamenti di Controlli Automatici, $5^a$ edizione, McGraw-Hill, Milano, 2025.
	\item[\textbf{-}] G. Calafiore, Appunti di controlli automatici, CLUT, Torino, 2006.
	\item[\textbf{-}] A. Isidori, Sistemi di Controllo, $2^a$ edizione, vol. primo, Siderea, Roma, 1992.
	\item[\textbf{-}] R. C. Dorf, R. H. Bishop, Modern Control Systems, $14^a$ edizione, Pearson Education, Upper Saddle River (U.S.A.), 2021.
\end{enumerate}

Riportiamo ora i link, tra di loro identici, per accedere a tutto il materiale messo a disposizione dai docenti.

\begin{enumerate}
	\item[\textbf{-}] \href{https://www.ladispe.polito.it/corsi/ContrAutoInf270/}{https://www.ladispe.polito.it/corsi/ContrAutoInf270/} 
	
	\item[\textbf{-}] \href{https://www.labinf.polito.it/materiale/ladispe/ContrAutoInf270/}{https://www.labinf.polito.it/materiale/ladispe/ContrAutoInf270/} 
	
\end{enumerate}

Nell'eventualità in cui il lettore in possesso di questo elaborato abbia individuato degli errori, inesattezze o voglia proporre delle migliorie non esiti a contattare il seguente indirizzo di posta elettronica: \href{mailto:francesco.abrami@studenti.polito.it?subject=Controlli Automatici - 18AKSOA&body= Buongiorno, 
scrivo riguardo al documento LaTex del corso di cui in oggetto.}{francesco.abrami@studenti.polito.it}

\subsection{Lezioni esercitazioni e laboratori}

Lezioni esercitazioni, come detto in precedenza, non hanno una suddivisione netta ma potrebbero essere mischiate tra di loro. Per quanto riguarda i laboratori verrà riservata loro una sezione dedicata allo svolgimento e soluzioni. Facciamo notare come le soluzione proposte possano non essere complete o non ottimali. 

Riteniamo corretto un confronto con le attività svolte in aula o laboratorio. 

Ricordiamo infine che per lo svolgimento delle esercitazioni di laboratorio verrà utilizzata la versione 2014a di Matlab e dei relativi pacchetti per motivi di compatibilità in sede d'esame.

\subsection{Prova d'esame}

La prova d'esame consta di una prova scritta della durata di circa tre ore. In particolare la prova scritta consta di due parti, ciascuna della durata di un'ora e mezza circa:

\begin{enumerate}
	\item[\textbf{-}] Nella prima parte, è necessario rispondere a dieci domande proposte con risposte a scelta multipla senza l'ausilio del calcolatore dove nel calcolo del punteggio è prevista una penalità per ogni risposta sbagliata.
	
	\item[\textbf{-}] Al termine della prima parte sono rese disponibili le stringhe delle risposte corrette. Sono ammessi a sostenere la seconda parte d'esame solo gli studenti che avranno risposto esattamente ad almeno sei delle dieci domande.
	
	\item[\textbf{-}] La seconda parte è costituita dal progetto di un controllore (da realizzare con l'ausilio di Matlab/Simulink) e da un esercizio breve su diagrammi polari e di Nyquist, stabilità ad anello chiuso o su controllori.
	
\end{enumerate}

Le due parti devono essere sostenute nello stesso appello. Se il punteggio conseguito in ciascuna delle due parti non è inferiore a 12/30, il voto finale in trentesimi è dato dalla media aritmetica non pesata dei punteggi conseguiti nelle due parti, altrimenti l'esame è verbalizzato con la dicitura "respinto".

Se il candidato si ritira durante la prova, l'esame è comunque verbalizzato con la dicitura "ritirato". Lo studente può ritirarsi anche dopo la prova, purché lo comunichi inviando una e-mail ad entrambi i docenti entro i termini e secondo le modalità comunicate in sede d'esame. 
Lo studente che non si ritira avrà il giudizio registrato come previsto dalle norme di legge.
In calce ai risultati di ciascun appello, sono riportate le modalità per avere informazioni in merito alla propria prova d'esame.

Vista la modalità di svolgimento dell'esame andiamo ora a vedere le regole generali per il suo svolgimento.

\begin{enumerate}
	\item[\textbf{-}] Durante gli esami è consentito avere sul tavolo solo una calcolatrice non programmabile, l'occorrente per scrivere, due fogli di appunti come più sotto specificato ed un moderato numero di fogli bianchi. Nessun altro materiale (appunti, libri, zaini, cellulari, smartwatch, palmari, computer portatili o altri dispositivi personali in grado di navigare in rete) è ammesso. 
	
	\item[\textbf{-}] Durante gli esami non è consentito l'uso di testi o appunti, eccezion fatta per un formulario costituito da due fogli formato A4 scritti su entrambe le facciate (un foglio da usarsi nella prima parte, l'altro nella seconda parte) su cui lo studente può riportare ogni nota egli ritenga utile, escludendo però: esercizi svolti in toto o in parte, risposte a esercizi specifici comunque codificate, porzioni di listati Matlab. Su tale formulario, strettamente personale, devono essere riportati chiaramente nome, cognome e matricola. E altresì concesso l'uso di materiale di supporto messo a disposizione dai docenti durante il corso: tavole delle trasformate di Laplace e Zeta, Carta di Nichols, diagrammi delle reti di compensazione.
	
	\item[\textbf{-}] Durante gli esami è consentito l'uso di calcolatrici che, oltre alle operazioni aritmetiche, prevedano funzioni trigonometriche dirette e inverse, logaritmi, esponenziali, radici, fattoriali, sommatorie, medie e altre funzioni statistiche. Non sono assolutamente ammesse calcolatrici programmabili o in grado di eseguire programmi predefiniti di qualunque tipo o con display grafici.
	
	\item[\textbf{-}] Gli studenti risultati in possesso di materiale non autorizzato oppure sorpresi a comunicare o a tentare di comunicare sono automaticamente bocciati.
	
\end{enumerate}

Il materiale indicato sopra come le varie carte e formulari possono essere trovati in fondo a questo elaborato nella sezione dedicata.












\end{document}