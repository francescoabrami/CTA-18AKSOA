\documentclass[../main.tex]{subfiles}

\graphicspath{{\subfix{../images/}}}

\begin{document}


\newpage

\section{Modellistica dei sistemi}

Viste ora tutte le varie introduzioni delle parti del corso entriamo nel vivo di ciò che andremo a trattare andando ad analizzare le varie tipologie di sistema ed introducendo il concetto stesso di sistema. 
Come per le successive parti del corso introdurremo alcuni degli esempi al fine di poter vedere un po' più in pratica i concetti trattati.

\subsection{Nozione di sistema}


Iniziamo introducendo la nozione di sistema dandone la definizione ed analizzandone le componenti e caratteristiche basilari.

\begin{definition}[\textbf{Nozione di sistema}]
\end{definition}	
Per \textbf{sistema} si intende un ente (fisico o astratto) dato dall'interconnessione di più parti elementari, per cui vale il principio di azione e reazione.

Riportiamo di seguito la rappresentazione grafica di un generico sistema $S$ caratterizzato da $u(\cdot)$ e da $y(\cdot)$ che altro non sono che l'ingresso (azione, causa) e l'uscita (reazione, effetto).\\

% FIGURA SISTEMA 

\begin{figure}[!ht]
\centering
\resizebox{0.5\textwidth}{!}{
\begin{circuitikz}
\tikzstyle{every node}=[font=\LARGE]
\draw  (8.75,11.75) rectangle (8.75,11.75);
\draw [ line width=0.8pt ] (3.75,12.75) rectangle  node {\LARGE \textbf{S}} (7.5,10.25);
\draw [line width=0.8pt, ->, >=Stealth] (7.5,11.5) -- (10,11.5)node[pos=0.5,above, fill=white]{$y(\cdot)$};
\draw [line width=0.8pt, ->, >=Stealth] (1.25,11.5) -- (3.75,11.5)node[pos=0.5,above, fill=white]{$u(\cdot)$};
\end{circuitikz} }
\label{fig:my_label}
\end{figure}

In maniera più generale possiamo dire che l'interesse del corso si focalizzi sulla variabile di uscita $y$ e del suo andamento influenzato dall'ingresso $u$.
Inoltre ci è possibile analizzare il comportamento di un sistema attraverso $S$ ovvero un insieme di relazioni matematiche (detto modello matematico) che legano fra loro l'ingresso $u$ e l'uscita $y$.\\

Infine, dati questi termini, ci è possibile evidenziare alcune problematiche di interesse nello studio dei sistemi:

\begin{enumerate}
	\item[\textbf{-}] \textbf{Previsione:} In questo primo caso, noti $u(\cdot)$ ed $S$, dobbiamo trovare l'uscita $y(\cdot)$ ovvero prevedere come il sistema si comporterà dato un certo ingresso noto. 
	
	\item[\textbf{-}] \textbf{Controllo:} In questo secondo caso, noti $y_{des}(\cdot)$ ed $S$, dobbiamo trovare l'ingresso $u(\cdot)$ tale per cui si ottenga in uscita $y_{des}(\cdot)$ ovvero una certa reazione desiderata a priori.
	
	\item[\textbf{-}] \textbf{Identificazione:} Nell'ultimo caso, noti $y(\cdot)$ ed $u(\cdot)$, dobbiamo trovare $S$ attraverso la sua identificazione. In altri termini dobbiamo capire, dati ingressi ed uscite, davanti a che tipologia di sistema ci troviamo.
\end{enumerate}

Infine, prima di concludere questa parte introduttiva, andiamo a dare alcune informazioni sulla notazione utilizzata fino ad ora e che troveremo all'interno di tutto il resto dell'elaborato.
In particolare si ha che:

\begin{enumerate}
	\item[\textbf{-}] \textbf{u($\cdot$)} e \textbf{y($\cdot$)}: Indicano le funzioni di ingresso ed uscita. Per esempio la forza applicata ad un corpo e la sua velocità nel tempo oppure l'andamento di tensioni e correnti all'interno di un determinato circuito.
	
	\item[\textbf{-}] \textbf{u($t$)} e \textbf{y($t$)}: Indicano i valori assunti dalle funzioni di ingresso ed uscita in un certo istante di tempo $t$. Queste scritture indicano valori istantanei, possiamo intenderli come valori misurati in un certo instante di tempo.
	
\end{enumerate}


\subsection{Sistemi statici e dinamici}

Introdotto ora il concetto di sistema e dati i dovuti cenni di notazione possiamo ora introdurre la prima grande differenza tra le tipologie di sistemi che incontreremo all'interno del corso ovvero quella tra sistemi statici e dinamici.

\begin{definition}[\textbf{Sistema statico}]
\end{definition}

Per sistema statico si intende un sistema il cui il legame ingresso-uscita è istantaneo o statico cioè il valore dell'uscita $y$ all'instante $t$ dipende esclusivamente dal valore dell'ingresso $u$ allo stesso instante $t$.

Possiamo descrivere quanto appena detto attraverso la seguente relazione:

\[y(t) = g\big(u(t)\big), \hspace{10pt} \forall t \]\

Un esempio banale di sistema statico è rappresentato da un resistore ideale di cui riportiamo lo schema di seguito.

% RESISTORE IDEALE


In particolare dato un certo instante di tempo la caduta di potenziale tra i suoi capi è data esclusivamente dalla corrente che lo attraversa nello stesso instante di tempo.
In termini matematici si ha che:

\[u(t) = i_R(t)\]
\[y(t) = V_R(t) = R \cdot i_R(t) =  g\big(u(t)\big), \hspace{10pt} \forall t \]\

Riportiamo infine 

\begin{definition}[\textbf{Sistema dinamico}]
\end{definition}	


\subsection{Classificazione dei sistemi}

\subsection{Esempi}







\end{document}